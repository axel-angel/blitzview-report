\documentclass[12pt]{beamer}
\usetheme{Berlin}

\usepackage[utf8x]{inputenc}
\usepackage[english]{babel}
\usepackage[T1]{fontenc}
\usepackage{lmodern}

\usepackage{amsmath}
\usepackage{amssymb}
\usepackage{enumerate}
\usepackage{graphicx}
\usepackage{cite}
\usepackage{listings}
\usepackage{hyperref}

\usepackage{lipsum}
\beamertemplatenavigationsymbolsempty

\pdfinfo{
    /Author (Axel Angel)
    /Title (Scala BlitzView)
    /Subject (Presentation)
    }

\title{Scala BlitzView \\Presentation}
\author{Axel Angel \\EPFL, Switzerland \\axel.angel@epfl.ch}
\date{June 10, 2014}

% "define" Scala listing
\lstdefinelanguage{scala}{%
  morekeywords={abstract,case,catch,class,def,%
    do,else,extends,false,final,finally,%
    for,if,implicit,import,match,mixin,%
    new,null,object,override,package,%
    private,protected,requires,return,sealed,%
    super,this,throw,trait,true,try,%
    type,val,var,while,with,yield},
  otherkeywords={=>,<-,<\%,<:,>:,\#,@},
  sensitive=true,
  morecomment=[l]{//},
  morecomment=[n]{/*}{*/},
  morestring=[b]",
  morestring=[b]',
  morestring=[b]"""
}
\newcommand{\eg}{e.\,g.\ }
\newcommand{\reff}[1]{~\ref{#1}}

\AtBeginSection[]
  {
     \begin{frame}<beamer>
     \frametitle{Plan}
     \tableofcontents[currentsection]
     \end{frame}
  }

\begin{document}
\begin{frame}
    \maketitle
\end{frame}

\begin{frame}
    \frametitle{Table of Contents}
    \tableofcontents
\end{frame}

\section{Introduction}
\begin{frame}
    \frametitle{Abstract}
    % 1/2 lines of desc
    Scala is a {\bf powerful} language which currently provides a built-in implementation for non-strict views with some important {\bf shortcomings} for the users such as unexpected and unintuitive behavior.
    %it's not a separate library, but a deeply intergrated part of standrard one(you have .view operation on every collection), the fact that you're able to do good job without requirement to integrate so deep is your advantage.

    In this work we created a new library, based on {\bf Scala Blitz}, to provide lightweight, non-strict and parallel-efficient collections.
    %collections? or just views
    We present the library API {\bf design}, implementation and how programmers can use and extend it.
\end{frame}

\begin{frame}
    \frametitle{What are Views?}
    \begin{definition}[View]
        A non-strict version of one or more collections.
    \end{definition}

    \begin{definition}[Non-strictness]
         Evaluation is postponed until the result is actually needed.
    \end{definition}
    Non-strictness allows lazy-evaluation for our View operations (transformers).
    Use-case: lots of successive operations on elements.
\end{frame}

\begin{frame}
    \frametitle{What do Views contain?}
    Views {\bf capture} operations (contract):
    \begin{itemize}
        \item $O(tn)$ memory, stack of transformers
        \item $O(n)$ operations: efficient computation in a single pass over the inner collection
    \end{itemize}
    where $t$ is the number of transformers and $n$ the number of elements.

    \begin{definition}[Forced View]
        The View is said to be {\it forced} when the postponed computations are performed over all the elements.
    \end{definition}
\end{frame}

\begin{frame}[fragile]
    \frametitle{Operations}
    Different type of operations:
    \begin{description}
        \item[Transformers:] Postponed and {\bf captured} in Views (no forcing). \\
            \eg \verb|map| and \verb|filter|.
            Type: \verb|[a] -> [b]|.
        \item[Folders:] Last operations, {\bf forcing}. \\
            \eg \verb|aggregate| and \verb|max|.
            Type \verb|[a] -> b|.
    \end{description}

    We focused on a subset of operations (on purpose!)
\end{frame}

\section{Previous works}
\begin{frame}
    \frametitle{Based on existing ideas}
    Learn from past works:
    \begin{description}
        \item[Scala Collections:] Lots of methods like {\tt flatMap} and data structures like {\tt List}, {\tt Array}, {\tt Map}, {\tt Set} with (im)mutable variants.
            Strict.
        \item[Scala Views:] Conversion from Scala Collections to Views, same interface. Immutable.
            Non-strict.
        \item[Java 8 Streams:] Like Scala Views, unlike Scala Streams. Non-strict. Semi-immutable (not transparent-referenceable\footnote{Can be replaced with its {\bf value} without changing the behavior}).
    \end{description}
\end{frame}

\begin{frame}
    \frametitle{What's good and what's not so good}
    Comparisons:
    \begin{description}
        \item[Scala Collections:] $+$Lots of methods, $+$Strict, $-$Non-parallel\footnote{Require explicit conversion {\tt Par}};
        \item[Scala Views:] $-$Lots of methods, $+$Non-strict, $-$Non-parallel\footnote{{\tt Par} has unexpected behavior};
        \item[Java 8 Streams:] $+$OK methods $+$Non-strict $-$Non-parallel\footnote{Require explicit conversion: {\tt parallelStream}} $-$Limitations of Generics\footnote{One per type: {\tt flatMapTo*}} $-$Poor usability.
    \end{description}
\end{frame}

\section{Design}
\begin{frame}
    \frametitle{Issues}
    What's problematic:
    \begin{itemize}
        \item Scala common interface inherited from Collections;
        \item Requiring explicit parallel conversion;
        \item Scala unexpected behavior between Views and Par;
        \item Java limitations and usability problems;
        \item Built-in vs ``Hackability'': extension.
    \end{itemize}
\end{frame}

\begin{frame}
    \frametitle{Solution}
    {\tt BlitzView}, or how we solved them:
    \begin{itemize}
        \item Scala common interface inherited from Collections \\
            $\Rightarrow$ Specific interface for Views
        \item Requiring explicit parallel conversion \\
        \item Scala unexpected behavior between Views and Par \\
            $\Rightarrow$ Built-in configurable parallelism (on operations)
        \item Java limitations and usability problems \\
            $\Rightarrow$ Scala type system and immutability
        \item Built-in vs ``Hackability'': extension \\
            $\Rightarrow$ Separate package and implicits conversions
    \end{itemize}
\end{frame}

\begin{frame}[fragile]
    \frametitle{Problem of interface}
    Scala common interface inherited from Collections\footnote{In fact {\tt Traversable}}:
    \begin{lstlisting}
    xs.view.map{x => println(x); x}.sorted
    // each number is printed
    // it returns a View
    \end{lstlisting}
    Other operations are {\bf forcing} like \verb|groupBy|, \verb|intersect|, \verb|sortBy| and the likes.
    %For example \verb|flatten| returns a {\bf List} (even with nested Views). % This is wrong!
\end{frame}

\begin{frame}[fragile]
    \frametitle{Problem of parallelism}
    Requiring explicit parallel conversion; \\
    unexpected behavior between Views and Par:\\

    \begin{lstlisting}
    val xs = (0 to 1000).par.view
    val ys = (0 to 1000).view.par
    \end{lstlisting}

    Which one is correct? Are they equivalent? In fact they are not, worse, the first loses its parallelism, while the second loses its non-strictness.
\end{frame}

\begin{frame}[fragile]
    \frametitle{Solution}
    \begin{lstlisting}[language=scala, basicstyle=\tiny]
trait BlitzView[+B] { // some have [A >: B] for covariance
  /* operators */
  def ++/:::(ys: BlitzView[B]): BlitzView[B]
  def ::(y: A): BlitzView[B]

  /* methods (transformers): V -> V (hidden: map + filter) */
  def flatMap[C, U](f: B => U)(implicit ctx: Scheduler, viewable: IsViewable[U, C]): BlitzView[C]

  /* methods: V -> other array structure */
  def toArray(implicit ct: ClassTag[A], ctx: Scheduler): Array[A]
  def toList(implicit ctx: Scheduler): List[A]

  /* methods: V -> V[constant type] */
  def toInts/Doubles/...(implicit f: Numeric[A]): BlitzView[Int]

  /* methods: V -> 1 (with Optional variants) */
  def aggregate[R](z: => R)(op: (A, R) => R)(reducer: (R, R) => R)(implicit ctx: Scheduler): R
  def reduce(op: (A, A) => A)(implicit ctx: Scheduler): A
  def min/max(implicit ord: Ordering[A], ctx: Scheduler): A
  def sum/product(implicit num: Numeric[A], ctx: Scheduler): A

  /* methods: V -> 1[constant type] */
  def size(implicit ctx: Scheduler): Int
  def count/find/exists/forall(p: B => Boolean)(implicit ctx: Scheduler): Int
}
    \end{lstlisting}
\end{frame}

\begin{frame}
    \frametitle{Remarks}
    Properties:
    \begin{itemize}
        \item Sufficient powerful methods implemented (but no more\footnote{Some says: ``less is more''})
        \item Implicits to configure parallelism (ScalaBlitz\footnote{Homepage: http://scala-blitz.github.io/} {\tt Scheduler})
        \item {\bf Modularity} using common implementation trait
        \item {\bf Extensibility} using implicits (supported collections)
    \end{itemize}
\end{frame}

\begin{frame}
    \frametitle{Modularity design}
    OOP Hierarchy:
    \begin{itemize}
        \item BlitzView trait (interface)
        \begin{itemize}
            \item BlitzViewImpl trait (interface+implementations)
             \\ subclasses implement {\tt >\,>} and {\tt genericInvoke}
            \begin{itemize}
                \item BlitzViewC(ollection)
                \item BlitzViewVV(iews)
                \item BlitzViewFlattenVs
                \item BlitzViewO(ption)
                \item BlitzViewS(ingleton)
                \item (Open to new implementations)
            \end{itemize}
        \item (Open to new design)
        \end{itemize}
    \end{itemize}
\end{frame}

\begin{frame}[fragile]
    \frametitle{Design Internals}
    \begin{lstlisting}[language=scala, basicstyle=\tiny]
/** BlitzView implementation with a single source par Collection. */
abstract class BlitzViewC[B] extends BlitzViewImpl[B] { self =>
  type A // type of source list
  val xs: Reducable[A] // source list
  def transform: ViewTransform[A, B] // stack of transforms

  override def >>[C](next: ViewTransform[B, C]) = new BlitzViewC[C] {
    type A = self.A
    val xs = self.xs
    def transform = self.transform >> next
  }

  override def genericInvoke[R](op: (B, ResultCell[R]) => ResultCell[R]
    , pstop: ResultCell[R] => Boolean)
    (reducer: (R, R) => R)(implicit ctx: Scheduler): ResultCell[R] =
  {
    val stopper = ViewUtils.toStopper(pstop)_
    xs.mapFilterReduce[R](transform.fold(op), stopper)(reducer)(ctx)
  }
}
    \end{lstlisting}
\end{frame}

\begin{frame}[fragile]
    \frametitle{Design Internals}
    \begin{lstlisting}[language=scala, basicstyle=\tiny]
trait ViewTransform[-A, +B] { self =>
  type Fold[A, F] = (A, ResultCell[F]) => ResultCell[F]

  def fold[F](g: Fold[B, F]): Fold[A, F]

  def >>[C](next: ViewTransform[B, C]) = new ViewTransform[A, C] {
    def fold[F](fd: Fold[C, F]): Fold[A, F] =
        self.fold(next.fold(fd))
  }
}

class Map[A, B](m: A => B) extends ViewTransform[A, B] {
  def fold[F](fd: Fold[B, F]): Fold[A, F] =
    (x, acc) => fd(m(x), acc)
}

class Filter[A](p: A => Boolean) extends ViewTransform[A, A] {
  def fold[F](fd: Fold[A, F]): Fold[A, F] =
    (x, acc) => if (p(x)) fd(x, acc) else acc
}
    \end{lstlisting}
\end{frame}

\begin{frame}
    \frametitle{Extensibility design}
    Different types of implicits:
    \begin{description}
        \item[Implicit-extension:] Automatic conversion to an extended ``proxy'' class under certain conditions.
            Useful to add methods ``on-the-fly'' depending on type shape.
            Our use-case: {\tt addFlatten}.
        \item[Implicit-evidence:] An implicit conversion that requires an (implicit) evidence to.
            One evidence per supported collection type.
            Our use-case: {\tt bview}.
    \end{description}
    Without changing code: easy to add methods to our design, add new supported collections (and View type).
    Disadvantage: less intuitive (``black-magic'' effect).
\end{frame}

\begin{frame}[fragile]
    \frametitle{Implicit-extension}
    An example of our use-case:
    \begin{lstlisting}[language=scala, basicstyle=\tiny]
  val v: BlitzView[BlitzView[Int]] =
    Array[Option[Int]](Some(1), None).map{_.bview}
  v.flatten // implicit-extension
    \end{lstlisting}

    Behind the scene:
    \begin{lstlisting}[language=scala, basicstyle=\tiny]
  implicit def addFlatten[U, B[_] <: BlitzView[_], C[_] <: BlitzView[_]](view: B[C[U]])
    = new ViewWithFlatten[U, B, C](view)

  class ViewWithFlatten[U, B[_] <: BlitzView[_], C[_] <: BlitzView[_]]
  (val view: B[C[U]]) extends AnyVal
  {
    def flatten()(implicit ctx: Scheduler, ct: ClassTag[U]): BlitzView[U] =
      new BlitzViewFlattenVs[U, B, C] {
        val zss = view
      }
  }
    \end{lstlisting}
\end{frame}

\begin{frame}[fragile]
    \frametitle{Implicit-evidence}
    An example of our use-case:
    \begin{lstlisting}[language=scala, basicstyle=\tiny]
  Array[Int](1, 2, ..., N).bview
    \end{lstlisting}

    Behind the scene:
    \begin{lstlisting}[language=scala, basicstyle=\tiny]
  implicit def toViewable[L, A](col: L)(implicit evidence: IsViewable[L, A]) =
    new Viewable[L, A](col)(evidence)

  trait IsViewable[L, A] {
    def apply(c: L): BlitzView[A]
  }

  class Viewable[L, A](col: L)(evidence: IsViewable[L, A]) {
    def bview: BlitzView[A] = evidence.apply(col)
  }

  /* all supported collections have an evidence */
  implicit def arrayIsViewable[T](implicit ctx: Scheduler) =
    new IsViewable[Array[T], T] {
      override def apply(c: Array[T]): BlitzView[T] = {
        View(c.toPar)(new Array2ZippableConvertor)
      }
    }

  /* ... */
    \end{lstlisting}
\end{frame}

\section{Conclusion}
\begin{frame}
    \frametitle{Future directions}
    There are more:
    \begin{itemize}
        \item Add new type of Views (Generators?)
        \item Extend API (keep least surprise)
        \item Generalize to other parallel libraries
        \item Add smart memorization
        \item Macro expansions (further optimization)
    \end{itemize}
\end{frame}

\begin{frame}
    \frametitle{Insights}
    What I have learned.
\end{frame}

\begin{frame}
    \frametitle{Thanks}
    {\Huge Thanks for your attention!}

    Prototype on GitHub:

    \url{https://github.com/axel-angel/scala-blitzview}

    Includes code samples for free! :)
\end{frame}

\end{document}
